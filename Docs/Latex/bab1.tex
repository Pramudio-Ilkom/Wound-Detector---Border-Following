%!TEX root = ./template-skripsi.tex
%-------------------------------------------------------------------------------
% 								BAB I
% 							LATAR BELAKANG
%-------------------------------------------------------------------------------

\chapter{PENDAHULUAN}

\section{Latar Belakang Masalah}

Kulit merupakan perlindungan pertama manusia dalam menjaga 
tubuhnya dari berbagai macam substansi. Apabila kulit 
terluka, maka diperlukan penanganan yang baik agar tidak 
terjadi infeksi.  Luka adalah keadaan di mana fungsi 
anatomis kulit normal mengalami kerusakan akibat proses 
patologis yang berasal dari internal maupun eksternal dan 
mengenai organ tertentu. Ketika luka tidak terinfeksi, 
maka pada normalnya luka tersebut akan melakukan 
penyembuhan. Proses penyembuhan terdapat beberapa fase, 
yaitu: hemostasis (beberapa jam pasca-terjadinya luka), 
inflamasi (1 - 3 hari), proliferasi (4 - 21 hari), dan 
remodelling (21 hari - 1 tahun). Fase-fase penyembuhan luka 
terjadi secara bertahap, namun dapat terjadi secara 
bersamaan (\textit{overlap}) (\cite{simon}).

Dari beberapa kondisi luka, terdapat luka yang proses 
penyembuhannya tidak normal dengan durasi fase-fase yang 
sesuai. Kondisi tersebut disebut dengan luka kronis 
(\cite{landen}), kondisi ini dapat memiliki kaitan dengan 
berbagai faktor yang memperlambat penyembuhan luka seperti 
adanya penyakit kronis, insufisiensi vaskuler, diabetes, 
gangguan nutrisi, penuaan, dan berbagai faktor lokal pada 
luka (tekanan, infeksi, dan edema). Secara umum, luka 
kronis dapat terjadi akibat ulkus vena, ulkus arteri, ulkus 
dekubitus, dan ulkus diabetik (\cite{zhao}).

Hal paling pertama yang dilakukan untuk menangani suatu 
masalah adalah mengidentifikasi masalah tersebut, luka 
tidak menjadi pengecualian. Pengidentifikasian luka yang 
akurat dapat membantu memberikan diagnosa yang akurat, 
penanganan luka yang tepat, memantau perbaikan luka, 
menghindari terjadinya komplikasi, serta dapat mengurangi 
biaya perawatan luka. Pengindentifikasian luka kronis 
umumnya didasarkan pada dua jenis tinjauan klinis, yaitu: 
tinjauan visual terhadap warna dominan luka dengan 
mengidentifikasi jaringan luka, dan tinjauan manual 
terhadap bentuk luka (area, perimeter, kedalaman, dan 
lain-lain) dengan pemeriksaan luka. Saat ini, tersedia 
dua teknik untuk pemeriksaan ini, yang pertama adalah 
metode manual langsung yang digunakan oleh dokter dan 
perawat untuk mengukur secara berkala dimensi luka 
menggunakan penggaris, dan teknik yang kedua adalah melacak 
batas luka pada kertas kalkir transparan yang ditempatkan 
pada kotak metrik. Kertas kalkir merupakan sebuah kertas 
yang memiliki permukaan yang tembus pandang dan sering 
digunakan oleh desainer untuk merancang desain atau gambar. 
Kertas ini memiliki struktur seperti kaca yang dapat 
dilihat secara tembus pandang ke bagian belakang kertas 
kalkir tersebut. Permukaan luka kemudian ditentukan dengan 
menghitung jumlah kotak secara manual setelah memindai 
gambar. Namun, kedua metode tersebut tidak dapat mengukur 
dimensi luka secara akurat dan kertas kalkir yang digunakan 
dapat menyebabkan infeksi pada luka (\cite{GuptaRTWS}). Salah 
satu metode yang diusulkan untuk mengatasi masalah ini 
adalah dengan menggunakan metode yang berbasis 
\textit{digital image processing}.

Metode berbasis \textit{digital image processing} merupakan 
alternatif untuk penilaian luka karena dapat memberikan 
langkah-langkah yang objektif, lebih akurat, dan dapat 
direproduksi. Salah satu keuntungan besarnya adalah metode 
ini memiliki resiko yang lebih rendah karena tidak ada kontak antara 
luka dan sistem pengukuran. Segmentasi citra merupakan 
proses mempartisi gambar digital menjadi beberapa segmen 
(set piksel, disebut juga \textit{superpixel}) yang 
memiliki fitur atau atribut yang sama. Segmentasi bertujuan 
untuk menyederhanakan atau mengubah representasi suatu 
citra menjadi sesuatu yang lebih bermakna dan lebih mudah 
untuk dianalisis. Segmentasi citra biasa digunakan untuk 
menemukan objek dan batas (seperti garis, kurva, dan 
lain-lain) dalam gambar (\cite{ShmmalaCBI}). Biasanya 
segmentasi menggunakan informasi lokal dalam gambar digital 
untuk menghitung segmentasi terbaik, seperti informasi 
warna yang digunakan untuk membuat histogram atau informasi 
yang mengindikasikan tepi, batas atau informasi tekstur 
(\cite{khattab}).

Segmentasi warna citra (\textit{color image segmentation}) 
didasarkan pada fitur warna piksel gambar yang 
mengasumsikan bahwa warna-warna homogen pada gambar 
bersesuaian dengan kelompok yang terpisah. Dengan kata 
lain, setiap kelompok mendefinisikan kelas piksel yang 
memiliki properti warna yang sama. Karena hasil segmentasi 
bergantung pada ruang warna (\textit{color space}) yang 
digunakan, tidak ada ruang warna tunggal yang dapat 
memberikan hasil yang dapat diterima untuk semua jenis 
gambar. Karena alasan ini, banyak penulis yang mencoba 
menentukan ruang warna yang sesuai dengan masalah 
segmentasi warna citra spesifik mereka (\cite{khattab}). 
Beberapa macam dari ruang warna (\textit{color space}), 
yaitu \textit{RGB, CMY(K), HSV, CIE, L*a*b, L*u*v,} dan 
\textit{YCrCb}. Setiap ruang warna (\textit{color space}) 
mempunyai sekurang-kurangnya 3 elemen warna dasar.

Salah satu cara mengidentifikasi luka adalah dengan 
tinjauan visual untuk identifikasi jaringan luka dari 
warna dominan. Warna luka akan memberikan banyak informasi 
penting berkaitan dengan perkiraan waktu penyembuhan luka, 
kondisi umum luka apakah dalam keadaan baik atau memburuk, 
dan risiko komplikasi. Warna tersebut dapat dipisahkan 
menjadi tiga warna, yaitu merah, kuning, dan hitam. 
Tinjauan visual ini, bisa dilakukan dengan metode 
segmentasi warna citra. Salah satu metode segmentasi warna 
citra yang telah dilakukan Aprilia Khairunnisa adalah 
dengan melihat ruang warna LAB terhadap segmentasi warna 
\textit{red, yellow, and black}. Proses segmentasi nya bisa 
dilakukan dengan menggunakan dua metode yaitu 
\textit{k-means} atau \textit{mean shift}. 
\textit{K-means} merupakan metode pengelompokan dari 
sejumlah \textit{cluster} yang terpisah. "K" mengacu pada 
jumlah \textit{cluster} yang ditentukan (\cite{YadavSeg}). 
Metode \textit{k-means} mempartisi \textit{n-set} input 
data menjadi \textit{k-cluster} di mana setiap set input 
data termasuk ke dalam cluster dengan mean terdekat 
(\cite{ZhengX}). Metode ini akan mempartisi data yang 
berkarakteristik sama ke dalam suatu kelompok yang sama dan 
data yang lainnya ke kelompok yang berbeda 
(\cite{Gustientiedina}). \textit{Mean shift} adalah teknik 
analisis \textit{non-parametic feature space} untuk mencari 
nilai maksimum dari fungsi kerapatan atau kepadatan yang 
diberikan dari data diskrit yang ada di fungsi tersebut. 
\textit{Mean shift} merupakan prosedur berulang (iteratif) 
sederhana yang menggeser setiap titik data ke rata-rata 
(\textit{mean}) titik data di daerahnya. Algoritma 
\textit{mean shift} disebut juga sebagai algoritma 
pencarian mode (\cite{ChengY}). 

Dari metode yang dilakukan oleh Aprilia Khairunnisa terdapat 
kekurangan, di mana hasil dari penelitiannya belum dapat 
memperlihatkan pengaruh dari penggunaan model warna LAB pada 
proses segmentasi (\cite{Aprilia}). ada satu hal yang masih 
dilakukan secara manual dalam proses metode tersebut, yaitu 
memasukkan hanya gambar luka saja, tanpa sekitar lukanya. 
Hal ini bisa dikembangkan dengan memasukan metode 
segmentasi yang mendeteksi tepi luka yaitu 
\textit{active contour}. 

Model \textit{active contour}, juga disebut \textit{Snake}, 
adalah \textit{framework} dalam pengolahan citra yang 
diperkenalkan oleh Michael Kass, Andrew Witkin, dan Demetri 
Terzopoulos untuk menggambarkan garis objek dari gambar 2D 
yang mungkin \textit{noisy}. Model \textit{active contour} 
populer dalam pengolahan citra, dan active contour banyak 
digunakan dalam aplikasi seperti \textit{object tracking}, 
\textit{shape recognition}, segmentasi, deteksi tepi, dan 
pencocokan stereo. \textit{Active contour} merupakan 
peminimalisir energi, dapat dideformasi yang terpengaruh 
oleh \textit{constraint} dan \textit{image forces} yang 
menarik ke arah objek kontur dan gaya di dalam yang menahan 
deformasi. \textit{Active contour} bisa diartikan sebagai 
model yang bisa dideformasi kepada citra dengan 
meminimalisir energi. Dalam dua dimensi, model bentuk aktif 
mewakili versi diskrit dari pendekatan ini, mengambil 
keuntungan dari model distribusi titik untuk membatasi 
rentang bentuk ke domain eksplisit yang dipelajari dari set 
pelatihan (\cite{kass}).

\emph{Active contour} merupakan metode yang biasanya tidak 
digunakan sendiri, dikarenakan metode ini perlu mengetahui 
bentuk suatu benda dan interaksi pengguna untuk mendapatkan 
hasil yang diinginkan. Pada Teknik yang dilakukan oleh 
Muhammad Rizki setelah memasukan citra dimulai dengan 
mengkonversi data \textit{RGB} menjadi \textit{grayscale}. 
Lalu gambar \textit{grayscale} tersebut mulai dideteksi 
menggunakan active contour di mana inisialisasi kurva 
awalnya yang seharusnya integer diubah dengan 
\textit{float}. Untuk rumusan energi internal, tidak diubah 
dari \textit{active contour} yang asli, tapi ketika masuk 
ke energi eksternal, persamaan yang dipakai berbeda sedikit. 
Setelah mendapatkan energi dari citra tersebut, dimulai 
proses update intersi kurva. Proses tersebut menggunakan 
memakai metode gradien arah direction untuk mendapatkan 
turunan pertama citra yang akan digunakan ke dalam rumus, 
lalu dilakukan iterasi yang dihitung melalui proses 
interpolasi (\cite{MRizki}).

Interpolasi dalam matematika umum adalah proses membuat 
suku-suku peralihan dari titik yang diketahui. Dalam 
pemrosesan gambar, biasanya digunakan untuk \textit{image scaling}, 
\textit{image resampling}, dan \textit{image resize}. Ada banyak algoritma 
yang saat ini digunakan untuk mengubah gambar digital. 
Kebanyakan dari mereka berupaya mereproduksi replika aslinya 
yang menarik secara visual. Sekarang seiring dengan 
teknologi untuk area tampilan yang lebih kecil untuk dilihat 
pada berbagai perangkat, ukuran gambar umumnya diambil 
sampelnya (atau disubsampel atau dikurangi) untuk 
menghasilkan thumbnail. Pengambilan sampel gambar 
(atau pembesaran atau interpolasi) paling umum dilakukan 
pada monitor atau televisi berukuran layar besar 
(\cite{Parsania}). Interpolasi yang akan dipakai penulis 
akan berdasarkan pada metode \textbf{NURBS} 
(\textit{Non-Uniform Rational B-Splines})(\cite{PiegTill96}) 
yang merupakan interpolasi penghalusan suatu kurva, ini 
bertujuan untuk menambahkan akurasi dalam pendeteksian luka, 
di mana pendeteksinya hanya menangkap tepi secara kasar.

seperti yang sudah tertuliskan dalam penilitian Muhammad 
Rizki, sebaiknya \textit{active contour} digantikan untuk 
meningkatkan akurasi deteksi luka (\cite{MRizki}). Dari metode 
yang dilakukan Muhammad Rizki, terdapat 
keunggulan di mana arah tepinya lebih jelas dibanding dengan 
citra asli sehingga membantu \textit{active contour} untuk 
melihat tepi suatu citra, tetapi ketika dilihat baik-baik, 
hasil dari gradien citra arah yang menggunakan interpolasi 
menambahkan arah tepi yang tidak diinginkan sehingga 
mengganggu pendeteksian \textit{active contour}. Untuk 
mengatasi masalah tersebut, penulis akan menggunakan 
algoritma \textit{Border Following} Suzuki.

Algoritma \textit{Border Following} 
suzuki merupakan salah satu algoritma topologi 
gambar biner digital yang pertama mendefinisikan hubungan 
hirarki antar pembatasan dan membedakan antara batas luar 
atau batas lubang.
Algoritma ini memindai gambar dari kiri ke kanan, mengecek 
ada nya objek piksel pada piksel yang dipindai. algoritma ini 
akan memindai sekitar piksel yang sedang dipindai untuk 
menentukan apakah piksel ini akan naik derajatnya. 
apabila sudah selesai pemindaian piksel ini, akan pindah 
pemindaian ke arah jarum jam, prioritas kanan. proses 
ini akan diulang sampai tidak ada piksel yang bisa digantikan
derajatnya. setelah sudah mendapatkan dari derajat dari 
semua piksel, maka bisa dilihat batas(\textit{border}) pada
gambar yang dipindai algoritma \textit{Border Following}(\cite{Suzuki}).

Penulis menginginkan untuk mengembangkan metode yang sudah 
pernah dilakukan oleh Muhammad Rizki(\cite{MRizki}). Penelitian ini 
bertujuan untuk meningkatkan akurasi dalam pendeteksian 
luka menggunakan \textit{active contour} dengan mengubah 
metode deteksi citra dengan \textit{border following}, lalu 
melakukan interpolasi pada hasil pendeteksian lukanya.
Penelitian ini bertujuan untuk membantu kalangan dokter dan 
perawat terkait penilaian luka kronis agar dapat memberikan 
hasil aproksimasi yang lebih dekat dengan \textit{ground
truth}(data asli yang diambil dari menggambar manual tepi luka).

\section{Rumusan Masalah}
Berdasarkan Latar belakang yang telah dikemukakan di atas, 
Fokus permasalahan pada penelitian ini adalah “Bagaimana 
mengembangkan pendeteksian keliling luka kronis menggunakan metode 
\textit{border following} yang dibantu dengan interpolasi spline”.

\section{Pembatasan Masalah}
Pembatasan masalah pada penelitian ini antara lain:
\begin{enumerate}
	\item Pendeteksian keliling luka kronis menggunakan 
	\textit{border following} yang dibantu dengan interpolasi 
	terhadap data citra luka yang 
	didapat dari penelitian luka Ns. Ratna Aryani, M.Kep, 
	tahun 2018 (\cite{Aryani:2018}).
	\item Penelitian dilakukan sampai mendapatkan hasil, 
	yaitu nilai akurasi dari selisih area kurva 
	\textit{border following} terhadap area 
	\textit{ground truth}(data asli yang diambil dari
	menggambar manual tepi luka).
\end{enumerate}

\section{Tujuan Penelitian}
Tujuan penelitian ini adalah untuk melanjutkan penelitian 
Rizki(\cite{MRizki}) dalam mendeteksi citra keliling luka dengan menggunakan metode 
\textit{Border Following} yang ditambahkan interpolasi spline
pada pendeteksi keliling luka kronis.

\section{Manfaat Penelitian}
\begin{enumerate}
	\item Bagi peneliti
		
	Penelitian ini merupakan media penerapan ilmu 
	pengetahuan, khususnya dalam pengembangan metode \textit{border following} 
	dalam pengkajian luka kronis serta membantu 
	penulis untuk menyelesaikan perkuliahan.
		
	\item Instansi Terkait
	 	
	Metode yang diajukan diharapkan dapat membuka peluang 
	untuk diajukan ke instansi kesehatan terkait dalam 
	proses pengkajian luka kronis.
	
	\item Bagi ilmu pengetahuan
	 	
	\begin{itemize}
		\item Mahasiswa
		
		Diharapkan penelitian ini dapat membantu dalam 
		penulisan paper yang bersangkutan dengan 
		pendeteksian luka.

		\item Bagi peneliti selanjutnya
		
		Diharapkan penelitian ini dapat dikembangkan oleh 
		peneliti selanjutnya untuk mengembangkan metode dari 
		penelitian ini.
	\end{itemize} 			
\end{enumerate}


% Baris ini digunakan untuk membantu dalam melakukan sitasi
% Karena diapit dengan comment, maka baris ini akan diabaikan
% oleh compiler LaTeX.
\begin{comment}
\bibliography{daftar-pustaka}
\end{comment}
