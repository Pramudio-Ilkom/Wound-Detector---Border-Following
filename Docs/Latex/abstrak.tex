\chapter*{\centering{\large{ABSTRAK}}}
\singlespacing{}

\textbf{PRAMUDIO}, Pendeteksian Keliling Luka Menggunakan Metode  
\textit{Border Following} dengan Bantuan Interpolasi Spline. Skripsi, Program Studi Ilmu Komputer, Fakultas Matematika dan Ilmu Pengetahuan Alam, Universitas Negeri Jakarta. juli 2024.
\\
\\
Luka kronis merupakan salah satu penyakit yang kompleks, khususnya bagi penyandang
penyakit Diabetes Melitus (DM). Proses penyembuhan luka diawasi oleh pekerja medis 
dengan asesmen, yaitu mengukur keliling luka dan melihat warna luka. Setelah asesmen, 
pekerja medis baru bisa memberi keputusan dalam penanganan luka. 
Proses penyembuhan luka perlu melakukan asesmen 
yang tepat dan pengelolaan yang efektif, hanya saja asesmen manual dalam pengukuran 
luka sangat memakan waktu dan berpotensi mengganggu penyembuhan. Skripsi ini 
bertujuan untuk membantu pekerja medis dalam asesmen luka kronis penggunaan 
metode \emph{border following} dibantu dengan interpolasi \textit{spline} 
dalam pemrosesan citra untuk mengurangi asesmen luka manual. 
Penelitian ini memiliki potensi untuk memberikan analisis yang 
objektif dan reliabel dalam penggunaan pengolahan citra untuk 
asesmen luka kronis. Dengan memanfaatkan teknologi pengolahan citra, 
peneliti mencoba mengatasi keterbatasan metode asesmen 
luka manual dengan menggunakan algoritma \emph{border following}. 
Yang pertama dilakukan dalam metode ini adalah mengambil foto luka dengan 
menggunakan perangkat seluler, dilanjutkan dengan pemotongan citra untuk 
meningkatkan akurasi, lalu metode \textit{border following} dijalankan pada 
foto yang sudah dipotong untuk mendapatkan daerah kurva sekitar luka, selanjutnya 
dihaluskan menggunakan interpolasi \textit{spline}. Metode ini dilakukan pada 
ketiga kategori luka; merah, kuning, dan hitam yang sebanyak 69 data citra. Eksperimen 
ini menunjukkan \textit{border following} hanya dapat mendeteksi 14 luka dengan 
rata-rata akurasi masing-masing; merah 70.9$\%$ dengan 4 luka, kuning 73.4$\%$ 
pada 2 luka, dan hitam 92.6$\%$ pada 8 luka
\\
\\
\textbf{Kata kunci:} luka kronis, \textit{border following}, 
interpolasi \textit{spline}, asesmen luka, pemrosesan citra, asesmen luka, penyembuhan luka
