%!TEX root = ./template-skripsi.tex
%-------------------------------------------------------------------------------
%                            	BAB IV
%               		KESIMPULAN DAN SARAN
%-------------------------------------------------------------------------------

\chapter{KESIMPULAN DAN SARAN}

\section{Kesimpulan}
Berdasarkan hasil experimen pengolahan citra, 
maka diperoleh kesimpulan sebagai berikut:

\begin{enumerate}
	\item Hasil deteksi luka dengan menggunakan 
	\textit{border Following} berhasil memindai 14 luka dari 69 
	luka di mana luka hitam yang paling banyak terpindai. Hasil 
	luka yang berhasil terpindai menggunakan \textit{border following} 
	menghasilkan rata-rata akurasi 84.3$\%$, dan setelah menggunakan 
	interpolasi rata-rata akurasi menjadi 82.9$\%$.
	
	\item Apabila dibandingkan dengan \textit{active contour} 
	yang digunakan Rizki, hasil pindai \textit{border Following} 
	lebih banyak gagal dibanding berhasil tetapi memiliki akurasi 
	yang lebih tinggi dengan rata-rata perbedaan 3.6$\%$ untuk 
	yang hanya menggunakan \textit{border Following} dan 1.3$\%$ 
	untuk yang dibantu dengan interpolasi.
\end{enumerate}

\section{Saran}
Adapun saran untuk penelitian selanjutnya adalah 
mengembangkan atau mengubah metode pendeteksian luka 
untuk meningkatkan keberhasilan deteksi luka dan 
meningkatkan akurasi deteksi luka


% Baris ini digunakan untuk membantu dalam melakukan sitasi
% Karena diapit dengan comment, maka baris ini akan diabaikan
% oleh compiler LaTeX.
\begin{comment}
\bibliography{daftar-pustaka}
\end{comment}
